\documentclass[conference]{IEEEtran}
\usepackage{cite}
\usepackage{amsmath,amssymb,amsfonts}
\usepackage{algorithmic}
\usepackage{graphicx}
\usepackage{textcomp}
\usepackage{xcolor}
\usepackage{color, soul} % For text highlighting

\begin{document}

\title{\Huge ECS 171 Group Project 1-Pager}

\author{
  \centering
  \IEEEauthorblockN{\Large Group 27}
  \IEEEauthorblockA{\large Jackson Cmelak, Brian Li, Meghana Manepalli, Susanna Mathew}
}

\maketitle

\large \textbf{Note: The group leader changed from Susanna Mathew to Meghana Manepalli}

\section{Problem Statement}

The average citizen of a developed country produces roughly 1,000 pounds of waste yearly, highlighting the importance of the waste management industry. From discarded packaging materials to everyday household items, waste management facilities identify, evaluate, and manage a variety of waste. Current waste management systems are outdated \cite{b1}, leading to the widespread, disorganized disposal of this waste, which poses a significant threat to the environment, public health, and our resource sustainability. The far-reaching consequences of mismanaged waste include land pollution, contamination of bodies of water, and depletion of natural resources.

A central problem in waste management is sorting all types of household waste. Existing methods of waste segregation often rely on human judgment are error-prone and inconsistent, leading to improper disposal and inadequate recycling rates. To address this issue, our project aims to develop a garbage classification system using machine learning techniques. 

If a robust and accurate garbage classification is successfully implemented, it can be used in many different fields. Some examples include:

\begin{itemize}
  \item Waste Management and Recycling    
    \begin{itemize}
      \item The model can be employed in waste sorting facilities to automate the categorization of waste into trash, compost, recyclables, and their subgroups.
    \end{itemize} 
  \item Environmental Conservation
    \begin{itemize}
      \item By identifying and quantifying waste in the environment, the model can monitor and model pollution levels, especially air, water, and land pollution.
    \end{itemize}
  \item Urban Planning
    \begin{itemize}
      \item Cities can optimize waste collection routes by using the model to track the amount and type of waste in different areas, leading to more efficient garbage collection systems. 
    \end{itemize}
  \item Education
    \begin{itemize}
      \item The model’s training and testing data can be used to educate individuals and communities on the importance of proper waste segregation and management. 
    \end{itemize}
\end{itemize}


\section{Dataset Description}

\section{Goals}

\section{Project Timeline}
\subsection{\textbf{Description and Background: 10/16 - 10/20}}
\begin{itemize}
    \item Complete "Introduction and Background" rough draft.
    \item Be able to describe the problem, significance, and methodology for a garbage classifier.
\end{itemize}
\subsection{\textbf{Literature Review and EDA: 10/20 - 10/27}}
\begin{itemize}
    \item Complete "Literature Review" and "Dataset Description and Exploratory Data Analysis" rough draft.
    \item Be able to describe different models and their pros and cons.
    \item Work on feature engineering, selection, and extraction necessary, as well as use data visualization and statistical techniques to better understand our dataset. 
\end{itemize}
\subsection{\textbf{Building Model and Finetuning: 11/03 - 11/17}}
\begin{itemize}
    \item Complete "Proposed Methodology and Experimental Results" rough draft.
    \item Build model(s), functions for validation, and continue to finetune model for accuracy.
\end{itemize}
\subsection{\textbf{Quality Assurance 11/17 - 11/24}}
\begin{itemize}
    \item Evaluate and test the model thoroughly. 
    \item Begin building the webpage.
\end{itemize}
\subsection{\textbf{Front End and Finishing Touches: 11/24 - 12/01}}
\begin{itemize}
    \item Complete the webpage.
    \item Complete "Conclusion and Discussion" section, ensure that "References" are complete.
    \item Complete any finishing touches on webpage, model, and project report.
\end{itemize}
\end{document}